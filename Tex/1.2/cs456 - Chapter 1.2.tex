\documentclass{article}

\usepackage{fullpage}
\usepackage{times}
\usepackage[utf8]{inputenc}
\usepackage[english]{babel}
\usepackage{enumitem}

\setlist{itemsep=2pt}

\begin{document}

\noindent
{CS 456 \hfill Hao Pan}\\
{Salahuddin, Mohammad}\\
{Spring 2018}

%%%%%%%%%%%%%%%%%%%%%%%%%%%%%%%

\begin{center}
\underline{\large \bf Chapter 1.2 - Network Edge}\\

{\it End systems, access networks, links.}\\

\noindent
{\hfill 03/05/2018 [Th]}
\end{center}

\underline{A Closer Look}

\begin{itemize}
\item A {\bf network edge} are the {\it hosts} (both clients and servers) and {\it servers} in a network.
\item An {\bf access network} or {\bf physical media} connects edges in a network. The connection could be either {\it wired} or {\it wireless}.
\item The {\bf network core} is a set of interconnected routers. It is essentially a {\it "network of networks"}. (?)
\item The end systems are connected to edge routers via:
\begin{itemize}
\item residential access networks
\item institutional networks (school, company, etc.)
\item mobile access networks (cellular data)
\end{itemize}
\item {\bf Bandwidth} is the data transfer rate, in {\it bits per second}, of a network.
\end{itemize}

\bigskip

\underline{\bf Access Networks}\\

\underline{Digital Subscriber Line (DSL)}

\begin{itemize}
\item Connecting to the central office DSLAM using existing {\it telephone lines}.
\item Voice and data are transmitted over the line at {\it different frequencies}. This technique is called {\bf Frequency Division Multiplexing} (FDM).
\item Transmission rates are {\it asymmetric} with:
\begin{itemize}
\item $<$ 10 Mbps upstream transmission rate
\item $<$ 25 Mbps downstream transmission rate
\end{itemize}
\end{itemize}

\underline{Cable Network}

\begin{itemize}
\item Data is transmitted through the television {\it cable lines}, using {\bf hybrid fiber coax} (HFC), to the ISP router.
\item FDM is, once again, used to transmit channels at different frequency bands.
\item A home shares access network to a {\it cable headend}.
\item Transmission rates are {\it asymmetric} with:
\begin{itemize}
\item $<$ 5 Mbps upstream transmission rate
\item $<$ 30 mbps downstream transmission rate
\end{itemize}
\end{itemize}

\newpage

\underline{Enterprise Access Networks (Ethernet)}

\begin{itemize}
\item It is typically used in companies, universities, etc.
\item End systems today usually connect into the Ethernet switch.
\item Transmission rates could be: 10 Mbps, 100Mbps, 1Gbps, 10Gbps.
\end{itemize}

\underline{Wireless Access Networks}

\begin{itemize}
\item A shared access network that wirelessly connects end systems to the router using an {\bf access point}.
\item {\bf Wireless LAN} connects devices {\it within a building} (~100 ft.) using 802.11 b/g/n WiFi.
\begin{itemize}
\item Transmission rates could be: 11, 54, or 450 Mbps.
\end{itemize}
\item {\bf Wide-area wireless access} is provided by {\it cellular operators} (10s km) using 3G, 4G: LTE.
\begin{itemize}
\item Transmission rates are between 1 - 10 Mbps.
\end{itemize}
\end{itemize}

\underline{\bf Physical Media}

\begin{itemize}
\item A {\bf bit} propogates between transmitter/receiver pairs.
\item A {\bf physical link} is a connection between the transmitter and the receiver.
\item In {\bf guided media}, signals are propagated in solid media such as {\it copper, fiber, coax} in a guided path.
\item In {\bf unguided media}, signals are propagated freely, such as with {\it radio}.
\end{itemize}

\underline{Cables}

\begin{itemize}
\item {\bf Twisted pair} cabling is a type of wiring where two insulated copper wires are {\it twisted together}.
\begin{itemize}
\item Category 5 cables transmit in 100Mbps, or 1Gbps Ethernet.
\item Category 6 cables transmit in 10Gbps.
\end{itemize}
\item {\bf Coxial cables} are two concentric copper conductors. Data can be transmitted in both directions. It is {\bf broadband}, so the cable has {\it multiple channels}.
\item {\bf Fiber optic cables} carry {\it light pulses} through glass fiber. Each pulse represents a bit. It is very high speed (10s-100s Gbps transmission rate), with a low error rate.
\end{itemize}

\underline{Radio}

\begin{itemize}
\item Signals are carried in {\it electromagnetic waves}, and could travel both ways.
\item Transmission could be affected by the environment, such as from reflection, physical objects, or other electromagnetic waves.
\item {\bf Terrestrial microwave} transmits in up to 45 Mbps channels.
\item {\bf LAN} (ex. WiFi) transmits in up to 54 Mbps.
\item {\bf Wide-area} (ex. cellular data) transmits at various speeds. 4G cellular transmits at ~10 Mbps.
\item {\bf Satellite} has Kbps - 45 Mbps channels. Transmissions have a 270 ms end delay.
\end{itemize}





\end{document}













