\documentclass{article}

\usepackage[utf8]{inputenc}
\usepackage{fullpage}
\usepackage{times}
\usepackage{tcolorbox}
\usepackage{enumitem}
\usepackage{multicol}
\usepackage{bookmark}

\bookmarksetup{
  numbered, 
  open,
}

\setlist{itemsep=2pt}
\renewcommand{\thesection}{Chapter 2.\arabic{section}}
\renewcommand{\thesubsection}{2.\arabic{section}.\arabic{subsection}}
\setcounter{section}{0}

\begin{document}

\noindent
{CS 456 \hfill Hao Pan}\\
{Salahuddin, Mohammad}\\
{Spring 2018}

%%%%%%%%%%%%%%%%%%%%%%%%%%%%%%%%%%%

\begin{center}
\section{Principles of Network Applications}
\noindent
{\hfill 22/05/2018 [T]}
\end{center}

\subsection{Example Network Apps}

\begin{multicols}{2}
\begin{itemize}
\item e-mail
\item web
\item text messaging
\item remote login
\item P2P file sharing
\item multi-user network games
\columnbreak
\item streaming stored video (YouTube, Netflix...)
\item voice over IP (Skype)
\item real-time video conferencing
\item social networking
\item searching
\item ...
\end{itemize}
\end{multicols}

\subsection{Creating a Network App}

\begin{itemize}
\item To create a network app, it must:
\begin{itemize}
\item Run on {\it different} end systems.
\item Communicate over a network.
\item {\bf Ex.} A web server software communicates with browser software.
\end{itemize}
\item There is no need to write software for {\it network-core devices} because:
\begin{itemize}
\item Network core devices do not run user applications.
\item Applications on end systems allow for rapid app development and propoation.
\end{itemize}
\item There are {\it two} possible structure of applications:
\begin{itemize}
\item Client-server
\item Peer-to-peer (P2P)
\end{itemize}
\end{itemize}

\subsection{Client-server Architecture}

\begin{itemize}
\item {\bf Servers}:
\begin{itemize}
\item Are an always-on host.
\item Have a permanent IP address.
\item Have data centers for scaling.
\end{itemize}
\item {\bf Clients}:
\begin{itemize}
\item Communicate with servers.
\item May or may not be intermittently connected.
\item Could have dynamic IP addresses.
\item Do not directly communicate with each other.
\end{itemize}
\end{itemize}

\clearpage

\subsection{P2P Architecture}

\begin{itemize}
\item {\it Not} an always-on server.
\item Random end systems directly communicate.
\item Peers {\it request} service from other peers and {\it provide} service to other peers in return.
\item Peers are not always connected and change IP addresses. Thus management is very complex.
\end{itemize}

\subsection{Processes Communicating}

\begin{itemize}
\item {\bf Process} refers to a program running within a host.
\item Within the {\it same host}, two processes communicate using inter-process communication.
\item Processes in different hosts communicate by exchanging messages.
\item In the client/server model, the client process initiates communication and the server process waits to be contacted.
\item {\bf Note:} Applications with P2P architectures have both client and server processes.
\end{itemize}

\subsection{Sockets}

\begin{itemize}
\item Processes send or receive messages to/from their sockets.
\item A {\bf socket} can be compared to a door:
\begin{itemize}
\item The sending process "shoves" a message out the door.
\item It relies on the {\it transport infrastructure} on the other side of the door to deliver the message to the socket at the receiving process.
\end{itemize}
\end{itemize}

\subsection{Addressing Processes}

\begin{itemize}
\item To receive messages, processes must have an {\it identifier}.
\item A host device has a unique 32-bit IP address
\item The host's IP address is not enough to identify the process because multiple processes can run on the same host.
\item Thus, the {\bf identifier} includes both the {\it IP address} and the {it port numbers} associated with the process on the host.
\item Example port numbers:
\begin{itemize}
\item HTTP server: 80
\item mail server: 25
\end{itemize}
\item {\bf Ex.} To send HTTP messages to the gaia.cs.umass.edu web server, use:
\begin{itemize}
\item IP address: 128.119.245.12
\item Port number: 80
\end{itemize}
\end{itemize}

\clearpage

\subsection{App-layer Protocol Defines}

\begin{itemize}
\item Types of messages exchanged (request, response...)
\item Message syntax (message fields)
\item Message semantics (Meaning of info in fields)
\item Rules for when and how process send/respond to messages.
\item Open protocols (HTTP, SMTP...)
\item Proprietary protocols (Skype..)
\end{itemize}

\subsection{Transport Services Needed by an App}

\begin{itemize}
\item Data Integrity
\begin{itemize}
\item Some apps (such as file transfers or web transactions) require 100\% {\it reliable} data transfers.
\item Other apps, like audio, can tolerate a bit of loss.
\end{itemize}
\item Timing
\begin{itemize}
\item Some apps (Internet, telephone, games...) are only {\it effective with low delay}.
\end{itemize}
\item Throughput
\begin{itemize}
\item Some apps (such as multimedia) require only a minimal amount of throughput to work effectively.
\item Some apps ("elastic apps") will use whatever throughput they can get.
\end{itemize}
\item Security
\begin{itemize}
\item Encryption, data integrity, etc. is important to ensure data does not get in the wrong hands.
\end{itemize}
\item Note: The table on slide 2-14 shows the importance of each service for different apps.
\end{itemize}

\subsection{Internet Transport Protocol Services}

\begin{multicols}{2}
\underline{\bf TCP Service:}
\begin{itemize}
\item {\it Reliable transport} between sending and receiving processes.
\item {\it Flow control} so the sender does not overwhelm the receiver.
\item {\it Congestion control} throttles the sender when the network is overloaded (so data is not lost).
\item {\bf Does not provide}: Timing, minimum throughput guarantee, or security.
\item {\it Connection oriented}, so setup is required between client and server processes.
\end{itemize}

\columnbreak

\underline{\bf UDP Service:}
\begin{itemize}
\item {\it Unreliable data transfer} between sending and receiving processes.
\item {\bf Does not provide}: Reliability, flow control, congestion control, timing, throughput guarantee, security, or connection setup.
\end{itemize}
\end{multicols}
{\bf Note}: The table on slide 2-16 shows whether TCP or UDP is preferred in different apps.

\clearpage

\subsection{Securing TCP}

\begin{itemize}
\item {\bf TCP \& UDP} have no encryption. Cleartext passwords sent into a socket {\it traverse in cleartext}.
\item {\bf SSL} provides encrypted TCP connection. SSL has data integrity and end-point authentication.
\item SSL is at the app layer. Apps use SSL libraries that communicate with TCP.
\item The SSL socket API encrypts passwords before sending them into the socket so they {\it traverse encrypted}.
\end{itemize}







\end{document}