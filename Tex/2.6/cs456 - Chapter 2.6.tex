\documentclass{article}

\usepackage[utf8]{inputenc}
\usepackage{fullpage}
\usepackage{times}
\usepackage{tcolorbox}
\usepackage{enumitem}
\usepackage{multicol}
\usepackage{bookmark}

\bookmarksetup{
  numbered, 
  open,
}

\setlist{itemsep=2pt}
\renewcommand{\thesection}{Chapter 2.\arabic{section}}
\renewcommand{\thesubsection}{2.\arabic{section}.\arabic{subsection}}
\renewcommand{\thesubsubsection}{2.\arabic{section}.\arabic{subsection}.\arabic{subsubsection}}
\setcounter{section}{5}

\begin{document}

\noindent
{CS 456 \hfill Hao Pan}\\
{Salahuddin, Mohammad}\\
{Spring 2018}

%%%%%%%%%%%%%%%%%%%%%%%%%%%%%%%%%%%

\begin{center}
\section{Video Streaming and Content Distribution Networks (CDNs)}
\noindent
{\hfill 07/06/2018 [Th]}
\end{center}

\subsection{Context}

\begin{itemize}
\item Video traffic is a major consumer of Internet bandwidth, with Netflix and YouTube taking up 37\% and 16\% of residential downstream traffic respectively.
\item With over 1 billion YouTube users and 75 million Netflix users, there is a challenge in how to reach so many users.
\item A single mega-video server would \emph{not} work because different users have \emph{different capabilities} (wired vs. mobile, bandwidth rich vs. bandwidth poor...). There are also other reasons which will be explained later on.
\item The solution is to use \emph{distributed application-level infrastructure}.
\end{itemize}

%%%%%%%%%%%%%%%%%%%%%%%%%%%%%%%%%%%
\subsection{Multimedia Video}

\begin{itemize}
\item A video consists of a sequence of images being displayed at a constant rate.
\item A digital image is made up of an array of pixels, where each pixel is represented by bits.
\item In practice, we use {\bf coding}: redundancy \emph{within} and \emph{between} images to decrease the number of bits used to encode an image.
\item There are \emph{two} types of coding:
\begin{itemize}
\item {\bf Spacial} (within image)
\begin{itemize}
\item {\bf Ex.} Instead of sending N values of the same color, send only two values: color value and number of repeated values.
\end{itemize}
\item {\bf Temporal} (from one image to the next)
\begin{itemize}
\item {\bf Ex.} Instead of sending a complete frame, only send \emph{differences} from previous frame.
\end{itemize}
\end{itemize}
\item {\bf CBR} (\emph{constant bit rate}) is when the video encoding rate is fixed.
\item {\bf VBR} (\emph{variable bit rate}) is when the video encoding rate changes as the amount of spatial and temporal coding changes.
\item {Ex.} {\it MPEG1 (CD-ROM) [1.5Mbps], MPEG2 (DVD) [3-6 Mbps], MPEG4 (often used in Internet) [<1 Mbps]}
\end{itemize}

%%%%%%%%%%%%%%%%%%%%%%%%%%%%%%%%%%%
\subsection{Streaming Multimedia: DASH}

\begin{itemize}
\item Stands for {\bf Dynamic Adaptive Streaming over HTTP}
\item The \emph{server} divides a video file into \emph{multiple chunks}, which are stored and encoded at different rates.
\item A {\bf manifest file} provides URLs for different chunks.
\item The \emph{client} periodically measures server-to-client bandwidth.
\item Consulting manifest, the client requests one chunk at a time and chooses the \emph{maximum coding rate sustainable} given its \emph{current bandwidth}. Note that different coding rates can be chosen at different points in time.
\item \emph{"Intelligence" at client} exists, where the client determines:
\begin{itemize}
\item \emph{When} to request chunks so that buffer starvation and overflow does not occur.
\item \emph{What encoding rate} to request, where higher quality rates are chosen when more bandwidth is available.
\item \emph{Where} to request chunks, so that it can request chunks from a URL server that's closer or has higher available bandwidth.
\end{itemize}
\end{itemize}

%%%%%%%%%%%%%%%%%%%%%%%%%%%%%%%%%%%
\subsection{Challenge}
\vspace{-3mm}
\emph{How to stream content, selected from millions of videos, to millions of simultaneous users?}

\subsubsection{Option 1}
\begin{itemize}
\item Use a single large "mega-server".
\item This method is not very reliable because:
\begin{itemize}
\item There is a single point of failure.
\item There is a single point of network congestion.
\item Clients far away have long paths.
\item Multiple copies of a video may be sent over an outgoing link.
\item Simply put, \emph{it does not scale}.
\end{itemize}
\end{itemize}

\subsubsection{Option 2}
\begin{itemize}
\item Store and serve multiple copies of videos at multiple geographically distributed sites (CDN)
\item {\bf Enter deep}: Pushing CDN servers into various access networks so that they're closer to users.
\item {\bf Bring home}: Smaller number (10s) of larger clusters in POPs near, but not within, access networks.
\end{itemize}

%%%%%%%%%%%%%%%%%%%%%%%%%%%%%%%%%%%
\subsection{A Closer Look}

\begin{itemize}
\item CDN stores copies of content at CDN nodes ({\bf ex.} Netflix stores copies of MadMen).
\item Subscribers request content from the CDN. The request is either directed to a nearby copy, which retrieves the content, or it is directed to a different copy if the network path is congested.
\item See example of steps of a video request on slide 2-96.
\item See example of how Netflix works on slide 2-97.
\end{itemize}

\end{document}