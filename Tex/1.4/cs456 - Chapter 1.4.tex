\documentclass{article}

\usepackage{fullpage}
\usepackage{times}
\usepackage[utf8]{inputenc}
\usepackage[english]{babel}
\usepackage{enumitem}

\usepackage{amsmath}

\setlist{itemsep=2pt}

\begin{document}

\noindent
{CS 456 \hfill Hao Pan}\\
{Salahuddin, Mohammad}\\
{Spring 2018}

%%%%%%%%%%%%%%%%%%%%%%%%%%%%%%%

\begin{center}
\underline{\large \bf Chapter 1.4 - Delay, Loss, and Throughput in Networks}\\
\noindent
{\hfill 10/05/2018 [Th]}
\end{center}

\underline{Loss and Delay}

\begin{itemize}
\item {\bf Loss and delay} can occur if packets arrive at a router faster than the router's output capacity.
\item There are {\it four} sources of packet delay ($d_{nodal} = d_{proc} + d_{queue} + d_{trans} + d_{prop}$):
\begin{itemize}
\item $d_{proc}$ is the {\bf nodal processing}. It checks the bits for errors and determines the output link. It is typically very fast ($<$ msec).
\item $d_{queue}$ is the {\bf queuing delay}. It is the time that must be waited at the output link {\it before transmission}. The delay depends on the router's congestion level.
\item $d_{trans}$ is the {\bf transmission delay}. It is equal to the packet length divided by the link bandwidth\\ ($d_{trans} = L/R$).
\item $d_{prop}$ is the {\bf propagation delay}. It is equal to the length of the physical link divided by the propagation speed ($d_{prop} = d/s$).
\end{itemize}
\item Car analogy: Slides 1-46 and 1-47.
\item A closer look at {\it queuing delay}, where {\it a} equals {\it average arrival rate}. If:
\begin{itemize}
\item La/R $\sim$ 0: the average queuing delay is very small.
\item La/R $\rightarrow$ 1: the average queuing delay grows larger.
\item La/R $>$ 1: the average queuing delay is infinite (more work arriving than can be serviced).
\end{itemize}
\item {\bf Packet loss} occurs when packets arrive at a link with a {\it finite buffer} and a {\it full queue}. When there is packet loss, additional packets are dropped.
\item The lost packet {\it may or may not} be retransmitted by the previous node or by the source end system.
\end{itemize}

\underline{Throughput}\\
\vspace{-4mm}

{\it Rate at which bits are transferred between the sender and receiver.}

\begin{itemize}
\item {\bf Instantaneous throughput} is the rate at a given point in time.
\item {\bf Average throughput} is the rate over a longer period of time.
\item A {\bf bottleneck link} is a link on the end of a path that constrains end-end throughput.
\item The bottleneck link can be found by looking for the link that is at capacity even though a link further down/upstream is not.
\end{itemize}


















\end{document}