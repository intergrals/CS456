\documentclass{article}

\usepackage[utf8]{inputenc}
\usepackage{fullpage}
\usepackage{times}
\usepackage{tcolorbox}
\usepackage{enumitem}
\usepackage{multicol}
\usepackage{bookmark}

\bookmarksetup{
  numbered, 
  open,
}

\setlist{itemsep=2pt}
\renewcommand{\thesection}{Chapter 5.\arabic{section}}
\renewcommand{\thesubsection}{5.\arabic{section}.\arabic{subsection}}
\setcounter{section}{0}

\begin{document}

\noindent
{CS 456 \hfill Hao Pan}\\
{Salahuddin, Mohammad}\\
{Spring 2018}

%%%%%%%%%%%%%%%%%%%%%%%%%%%%%%%%%%%

\begin{center}
\section{Introduction}
\noindent
\end{center}

\subsection{Network-Layer Functions}

\begin{itemize}
\item The network-layer has two functions:
\begin{itemize}
\item {\bf Forwarding}: to move packets from the router's input to the appropriate router output. It is in the \emph{data plane}.
\item {\bf Routing}: to determine the route taken by packets from source to destination. It is in the \emph{control plane}.
\end{itemize}
\item There are two approaches to structuring the network control plane:
\begin{itemize}
\item Per-router control. This is the traditionally used method.
\item Logically centralized control, which is software defined networking.
\end{itemize}
\item In the {\bf per-router control plane}, individual routing algorithm components in every router interact with each other in the control plane to compute forwarding tables.
\item In the {\bf logically centralized control plane}, a distinct controller interacts with the local control agents in routers to compute forwarding tables. This controller is usually remotely located.
\end{itemize}

\end{document}