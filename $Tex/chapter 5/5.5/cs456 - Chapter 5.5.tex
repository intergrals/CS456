\documentclass{article}

\usepackage[utf8]{inputenc}
\usepackage{fullpage}
\usepackage{times}
\usepackage{tcolorbox}
\usepackage{enumitem}
\usepackage{multicol}
\usepackage{bookmark}

\usepackage[]{algorithm2e}

\bookmarksetup{
  numbered, 
  open,
}

\setlist{itemsep=2pt}
\renewcommand{\thesection}{Chapter 5.\arabic{section}}
\renewcommand{\thesubsection}{5.\arabic{section}.\arabic{subsection}}
\setcounter{section}{4}

\begin{document}

\noindent
{CS 456 \hfill Hao Pan}\\
{Salahuddin, Mohammad}\\
{Spring 2018}

%%%%%%%%%%%%%%%%%%%%%%%%%%%%%%%%%%%

\begin{center}
\section{The SDN Control Plane}
\noindent
\end{center}

\subsection{Software Defined Networking (SDN)}

\begin{itemize}
\item The Internet network layer has historically been implemented via a distributed, per-router approach. A router contains switching hardware and runs implementations of Internet standard protocols (such as \emph{IP, RIP, IS-IS, OSPF, BGP...}) in the router OS. There are different sections for different network layer functions (such as \emph{firewall, load balancer, NAT boxes...}). In around 2005, there was interest in rethinking the network control plane.
\item Recall that in a {\bf per-router control plane}, individual routing algorithm components in every router interact with each other in the control plane to compute forwarding tables.
\item Recall that in a {\bf logically centralized control plane}, a distinct controller interacts with local control agents in routers to compute forwarding tables, and that the controller is usually remotely located. This is how SDN operates.
\item A logically centralized control plane:
\begin{itemize}
\item Provides easier network management. Router misconfigurations are avoided and there is greater flexibility of traffic flows.
\item Table-based forwarding (OpenFlow) allows "programming" routers. {\bf Centralized programming}, which computes tables centrally and then distribute, is the easier option. {\bf Distributed programming} is the more difficult implementation, which computes tables as a result of a distributed algorithm implemented in every router.
\item It is an open implementation of the control plane.
\end{itemize}
\end{itemize}

%%%%%%%%%%%%%%%%%%%%%%%%%%%%%%%%%%%
\subsection{SDN Controller}

\begin{itemize}
\item The {\bf SDN controller} maintains network state information.
\item It interacts with \emph{network control applications} "above" using {\bf northbound API}. The API allows network-control applications to read and write network state and flow tables in the state-management layer. Applications can register to receive notification when a state-change occurs.
\item It interacts with \emph{network switches} "below" using {\bf southbound API}. It is communication between the controller and the controlled devices.
\item It is implemented as a distributed system for performance, scalability, fault-tolerance, and robustness.
\item See a diagram of the SDN controller relative to its northbound and southbound apps on {\bf slide 5-63}.
\item {\bf Data plane switches} are fast and simple switches that implement generalized data-plane forwarding (packet-handling rules) in the hardware. The flow table is computed and installed by the controller, and an API is used for table-based switch control (like OpenFlow). It is a protocol for communicating with the controller (like OpenFlow).
\item {\bf Control applications} are the \emph{"brains" of control}. They implement control functions using lower-level services and API provided by the SDN controller. They can be provided by a 3rd party, where they're distinct from routing vendors or SDN controllers. In such a case, they are \emph{unbundled}.
\end{itemize}

%%%%%%%%%%%%%%%%%%%%%%%%%%%%%%%%%%%
\subsection{Components of SDN Controller}

\begin{itemize}
\item The {\bf interface to the network-control application layer} has network control apps. It provides different types of APIs, such as ones for abstraction or for communication.
\item The {\bf network-wide state-management layer} contains up-to-date info about the state of networks, links, switches, and services.
\item The {\bf communication layer} communicates between the SDN controller and network controlled devices.
\item See an example of interaction between the different planes on {\bf slide 5-66 and 5-67}.
\item There are challenges set to improve SDN by:
\begin{itemize}
\item \emph{Hardening the control plane} so that it's more dependable, reliable, and performance-scalable. It also doesn't hurt to be more secure.
\item Networks and protocols \emph{meeting mission-specific requirements}, such as making it real-time, ultra-reliable, or ultra-secure in specific situations.
\item \emph{Scale it} proportional to the Internet (which would actually be pretty hard to do).
\end{itemize}
\end{itemize}

\end{document}