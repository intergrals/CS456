\documentclass{article}

\usepackage[utf8]{inputenc}
\usepackage{fullpage}
\usepackage{times}
\usepackage{tcolorbox}
\usepackage{enumitem}
\usepackage{multicol}
\usepackage{bookmark}

\bookmarksetup{
  numbered, 
  open,
}

\setlist{itemsep=2pt}
\renewcommand{\thesection}{Chapter 6.\arabic{section}}
\renewcommand{\thesubsection}{6.\arabic{section}.\arabic{subsection}}
\setcounter{section}{1}

\begin{document}

\noindent
{CS 456 \hfill Hao Pan}\\
{Salahuddin, Mohammad}\\
{Spring 2018}

%%%%%%%%%%%%%%%%%%%%%%%%%%%%%%%%%%%

\begin{center}
\section{Multiple Access Protocols}
\noindent
\end{center}

\subsection{Multiple Access Links and Protocols}

\begin{itemize}
\item There are two types of "links":
\begin{itemize}
\item {\bf Point-to-point} is used for dial-up access. It is a link between the Ethernet switch and the host.
\item {\bf Broadcast} is used for old-fashioned Ethernet.
\end{itemize}
\item {\bf Multiple access protocols} use a single shared broadcast channel. There may be {\bf interference} when two or more nodes transmit simultaneously. If a node receives two or more signals at the same time, there is {\bf collision}.
\item It is a distributed algorithm that determines how nodes share a channel (for example, determining when a node can transmit). Communication about channel sharing must be about the channel itself, meaning they can't try to coordinate with out-of-band channels.
\item See an ideal multiple access protocol example on {\bf slide 5-13}.
\end{itemize}

%%%%%%%%%%%%%%%%%%%%%%%%%%%%%%%%%%%
\subsection{MAC Protocols: Taxonomy}

\subsubsection{Three Classes}
\begin{itemize}
\item There are three broad classes:
\begin{itemize}
\item {\bf Channel partitioning} divides a channel into smaller pieces (time slots, frequency, code...) and allocates a piece to each node for exclusive use.
\item {\bf Random access} does not divide channels and allows collisions, but it has procedures to "recover" from collisions.
\item {\bf "Taking turns"} allows nodes to take turns; however, nodes with more to send are permitted to take longer turns.
\end{itemize}
\end{itemize}

\subsubsection{Channel Partitioning: TDMA}
\begin{itemize}
\item {\bf Time Division Multiple Access}
\item Access to channels are provided in \emph{"rounds"}.
\item Each station gets a fixed length slot in each round. The $length = packet$ $transmission$ $time$.
\item Unused slots go to idle.
\item For example, in a 6-station LAN, if stations 1, 3, and 4 have packets, then slots 2, 5, and 6 are idle ({\bf slide 5-15}).
\end{itemize}

\clearpage

\subsubsection{Channel Partitioning: FDMA}
\begin{itemize}
\item {\bf Frequency Division Multiple Access}
\item The channel is divided into \emph{frequency bands}.
\item Each station is assigned a fixed frequency band.
\item Unised transmission time in frequency bands are left idle.
\item For example, in a 6-station LAN, if stations 1, 3, and 4 have packets, then frequenc bands 2, 5, and 6 are idle ({\bf slide 5-16}).
\end{itemize}

\subsubsection{Random Access Protocols}
\begin{itemize}
\item When a node has a packet to send, it will be transmitted at the full channel data rate, $R$.
\item There is no prior coordination between nodes at all.
\item If two or more nodes are transmitting at the same time, a \emph{collision} occurs.
\item {\bf Random access MAC protocol} specifies both how to detect collisions and how to recover from them (such as using delayed retransmissions).
\item Examples of random access MAC protocols are: \emph{slotted ALOHA, ALOHA, CSMA, CSMA/CD, CSMA/CA}.
\end{itemize}

\subsubsection{Carrier Sense Multiple Access (CSMA)}
\begin{itemize}
\item CSMA listens before transmission. If it senses that the \emph{channel is idle}, then the entire frame is transmitted. If the \emph{channel is busy}, then the transmission is deferred to a later time.
\item Note that collisions can still occur if two nodes do not hear each other's transmission. In such a case, the entire packet transission time is wasted.
\end{itemize}

\subsubsection{CSMA/CD (Collision Detection}
\begin{itemize}
\item The carrier sensing and deferral work exactly as in CSMA.
\item Collisions are detected within a shrot time, and colliding transmissions are aborted so the channel isn't wasted as much.
\item Collision detection works differently in wired and wireless LANs:
\begin{itemize}
\item It is easy in wired LANs. The signal strength is measured and compared to the transmitted and received signals.
\item It is much harder in wireless LANS. Received signal strength is overwhelmed by local transmission strength.
\end{itemize}
\item See CSMA/CD algorithm on {\bf slide 5-22}.
\item CSMA/CD has an efficiency of: $\frac{1}{1+5t_{prop}/t_{trans}}$. $t_{prop}$ is the max prop delay between 2 nodes in LAN, and $t_{trans}$ is the time to transmit max-size frame.
\end{itemize}

\clearpage

\subsubsection{CSMA/CA (Collision Avoidance)}
\begin{itemize}
\item This protocol method tries to avoid collisions that occur when two or more nodes transmit at the same time.
\item Collision detection dos not exist in 802.11 because it is difficult to sense collisions when transmitting due to weak received signals that fade (since collision sensing is received).
\end{itemize}

\subsubsection{"Taking Turns"}
\begin{itemize}
\item The taking turns protocol has two variations.
\item In a {\bf polling protocol}, a designated master node "invites" slave nodes to transmit turn-by-turn.
\begin{itemize}
\item This is usually used with "dumb" slave devices. They are told the max number of frames they are permitted to transmit.
\item The master node detects that the slaves are done by a lack of activity. They will then cycle to the next node.
\item There are concerns with \emph{polling overhead}, \emph{latency}, and \emph{a single point of failure at the master node}.
\end{itemize}
\item Another {\bf token passing} protocol has no master node, but rather tokens being exchanged between nodes.
\begin{itemize}
\item Tokens are always exchanged between nodes in a fixed order.
\item A node holds a token while transmitting up to a maximum number of frames or until it has nothing left to transmit. Then it will pass the token to the next node.
\item It has the similar concerns as polling: \emph{token overhead}, \emph{latency}, and \emph{a single point of failure at the node holding a token}.
\end{itemize}
\end{itemize}

\subsubsection{Comparison}
\begin{itemize}
\item {\bf Channel partitioning MAC protocols} share channels efficiently and fairly, even at high loads, but it is not very efficient at low load since there is delay in channel access. $I/N$ bandwidth is allocated even if there is only 1 active node. There is \emph{Time Division} and \emph{Frequency Division}.
\item {\bf Random access MAC protocols} are efficient at low load. A single node can fully utilize a channel, but at high load, there is a high risk of collision overhead. \emph{CSMA/CD} is used in Ethernet while \emph{CSMA/CA} is used in 802.11.
\item The {\bf "taking turns" protocol} looks for the best of both worlds. It could use \emph{polling} from a central site, or \emph{token passing}. Bluetooth, FDDI, and token rings use this protocol method.
\end{itemize}



















\end{document}