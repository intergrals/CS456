\documentclass{article}

\usepackage[utf8]{inputenc}
\usepackage{fullpage}
\usepackage{times}
\usepackage{tcolorbox}
\usepackage{enumitem}
\usepackage{multicol}
\usepackage{bookmark}

\bookmarksetup{
  numbered, 
  open,
}

\setlist{itemsep=2pt}
\renewcommand{\thesection}{Chapter 2.\arabic{section}}
\renewcommand{\thesubsection}{2.\arabic{section}.\arabic{subsection}}
\setcounter{section}{1}

\begin{document}

\noindent
{CS 456 \hfill Hao Pan}\\
{Salahuddin, Mohammad}\\
{Spring 2018}

%%%%%%%%%%%%%%%%%%%%%%%%%%%%%%%%%%%

\begin{center}
\section{Web and HTTP}
\noindent
{\hfill 24/05/2018 [Th]}
\end{center}

\subsection{Review of Web and HTTP}

\begin{itemize}
\item {\it Web pages} are made up of {\it objects}.
\item An {\bf object} can be an {\it HTML file, a JPEG image, a Java applet, an audio file, etc...}
\item A web page consists of a base HTML file, which references several objects, which are addressable by {\it a URL}.
\end{itemize}

%%%%%%%%%%%%%%%%%%%%%%%%%%%%%%%%%%%
\subsection{Overview of HTTP}

\begin{itemize}
\item {\bf HTTP} stands for {\it Hypertext Transfer Protocol}.
\item It uses a {\it client/server model}:
\begin{itemize}
\item The {\bf client} is the browser that requests, receives, and displays web objects.
\item The {\bf server} is the web server, which sends objects in response to requests.
\end{itemize}
\item TCP is used by the client to create a socket connection to the server on port 80. The server will accept the TCP connection and then HTTP messages will be exchanged between the client browser and the web server. When it is finished, the connection is closed.
\item Note that {\it HTTP is "stateless"}, meaning that the server does not keep information about past client requests.
\item There are {\it two types} of HTTP connections:
\begin{itemize}
\item In {\bf non-persistent HTTP}, at most one object can be sent over the TCP connection before the connection is closed. Thus, downloading multiple objects would require multiple connections.
\begin{itemize}
\item It has a response time of 2RTT + file transmission time (1 RTT to initiate TCP connection, and one for HTTP request).
\item Browsers often open multiple TCP connections in parallel to fetch referenced objects.
\end{itemize}
\item In {\bf persistent HTTP}, multiple objects can be sent over a single TCP connection between client and server.
\begin{itemize}
\item Very low response time (as little as 1 RTT + transmission time) because the connection is left open.
\end{itemize}
\end{itemize}
\item {\bf Note}: {\bf RTT} is the time for a small packet to travel from the client to the server and back.
\item There are two types of HTTP messages: {\it request}, and {\it response}.
\item See slide 2-28 to 2-31 for the format of request messages and response messages.
\item HTTP/2 was released on November 2012. It uses a new method to decrease latency and increase page load speeds in web browsers.
\begin{itemize}
\item In HTTP/2, HTTP messages are broken down into independent frames. They are interleaved and reassembled after being received.
\item Messages are interleaved in parallel, so none are blocked.
\item A single connection can also deliver multiple requests and responses in parallel, which means multiple connections are no longer needed for parallel delivery.
\end{itemize}
\end{itemize}

\clearpage

%%%%%%%%%%%%%%%%%%%%%%%%%%%%%%%%%%%
\subsection{Cookies}

\begin{itemize}
\item Many websites use cookies for {\it 4 components}:
\begin{enumerate}
\item Cookie header line of HTTP response message.
\item Cookie header line in next HTTP request message.
\item Cookie file kept on user's host, which is managed by the user's browser.
\item Back-end databases of websites.
\end{enumerate}
\item As an {\bf example}:
\begin{itemize}
\item If Susan always accesses the internet from PC, and she visits a website for the first time, when the HTTP requests arrives at that website, the website will create a {\it unique ID} and an {\it entry in backend database for the ID}.
\end{itemize}
\item Cookies can be used for:
\begin{itemize}
\item authorization
\item shopping carts
\item recommendations
\item user session state
\end{itemize}
\end{itemize}

\subsection{Web Caches}
\vspace{-3mm}
{\it Also known as a "proxy server"}

\begin{itemize}
\item The goal of a {\bf web cache} is to fulfill the client request {\it without involving the origin server}.
\item Browsers usually accesses the web via a cache. When the browser sends HTTP requests, they are sent to the cache.
\begin{itemize}
\item If the object is in the cache, then the cache {\it returns the object}.
\item Otherwise, the cache {\it requests the object} from the origin server, then returns it to the client.
\end{itemize}
\item A cache acts as both a {\it client} and a {\it server}.
\item The cache is typically installed by the ISP.
\item Web caching helps to:
\begin{itemize}
\item Reduce response time for client requests.
\item Reduce traffic on an institution's access link.
\end{itemize}
\item To calculate total delay, do:\\
(1 - cache satisfaction frequency) * (delay from origin server) + (cache satisfaction frequency) * (delay when satisfied by cache)
\item A {\bf conditional GET} doesn't send objects if cache already has an up-to-date cached version.
\begin{itemize}
\item There will be no object transmission delay.
\item The cache {\it specifies the date} of the cached copy in the HTTP request.
\item The server response contains {\it no object} if the cached copy is up-to-date.
\end{itemize}
\end{itemize}
















\end{document}